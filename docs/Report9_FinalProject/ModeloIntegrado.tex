\documentclass[final,5p]{elsarticle}

% \documentclass[preprint,12pt]{elsarticle}

%% Use the option review to obtain double line spacing
%% \documentclass[authoryear,preprint,review,12pt]{elsarticle}

%% Use the options 1p,twocolumn; 3p; 3p,twocolumn; 5p; or 5p,twocolumn
%% for a journal layout:
% \documentclass[final,1p,times]{elsarticle}
%% \documentclass[final,1p,times,twocolumn]{elsarticle}
% \documentclass[final,3p,times]{elsarticle}
%% \documentclass[final,3p,times,twocolumn]{elsarticle}
% \documentclass[final,5p,times]{elsarticle}
%% \documentclass[final,5p,times,twocolumn]{elsarticle}
\usepackage[portuguese]{babel}

%% For including figures, graphicx.sty has been loaded in
%% elsarticle.cls. If you prefer to use the old commands
%% please give \usepackage{epsfig}

%% The amssymb package provides various useful mathematical symbols
\usepackage{amssymb}
\usepackage{amsmath}
\usepackage{multirow}
\usepackage{tabularx}

\usepackage{pgfplots}
\pgfplotsset{compat=1.18}
\usepgfplotslibrary{statistics}
\usepackage{pgfplotstable}

\usepackage{placeins}
\usepackage{hyperref}
\numberwithin{equation}{section}

\usepackage{algorithm}
\usepackage[noEnd=true, indLines=true]{algpseudocodex}
\algrenewcommand\algorithmicrequire{\textbf{Entrada:}}
\algrenewcommand\algorithmicwhile{\textbf{Enquanto}}
\algrenewcommand\algorithmicrepeat{\textbf{Repete}}
\algrenewcommand\algorithmicuntil{\textbf{Até}}
\algrenewcommand\algorithmicif{\textbf{Se}}
\algrenewcommand\algorithmicthen{\textbf{então}}
\algrenewcommand\algorithmicelse{\textbf{Caso contrário}}
\algrenewcommand\algorithmicensure{\textbf{Objetivo:}}
\algrenewcommand\algorithmicreturn{\textbf{Retorna:}}
\algrenewcommand\algorithmicdo{\textbf{faça}}
\algrenewcommand\algorithmicforall{\textbf{Para todos}}
\algnewcommand{\LineComment}[1]{\State \(\triangleright\) \textcolor{black!50}{\emph{#1}}}

\newcommand*{\squareb}{\textcolor{black}{\rule{0.5em}{0.5em}}}
\newcommand*{\squareg}{\textcolor{gray}{\rule{0.5em}{0.5em}}}

% \usepackage[fleqn]{nccmath}
% \usepackage{multicol}


%=========== Gloabal Tikz settings
% \pgfplotsset{compat=newest}
% \usetikzlibrary{math}
% \pgfplotsset{
%     height = 10cm,
%     width = 10cm,
%     tick pos = left,
%     legend style={at={(0.98,0.30)}, anchor=east},
%     legend cell align=left,
%     }
%  \pgfkeys{
%     /pgf/number format/.cd,
%     fixed,
%     precision = 1,
%     set thousands separator = {}
% }

%% The amsthm package provides extended theorem environments
%% \usepackage{amsthm}

%% The lineno packages adds line numbers. Start line numbering with
%% \begin{linenumbers}, end it with \end{linenumbers}. Or switch it on
%% for the whole article with \linenumbers.
%% \usepackage{lineno}

\usepackage{listings}
\usepackage{xcolor}

\definecolor{codegreen}{rgb}{0,0.6,0}
\definecolor{codegray}{rgb}{0.5,0.5,0.5}
\definecolor{codepurple}{rgb}{0.58,0,0.82}
\definecolor{backcolour}{rgb}{0.98,0.98,0.98}

\lstdefinestyle{mystyle}{
    backgroundcolor=\color{backcolour},
    commentstyle=\color{codegreen},
    keywordstyle=\color{magenta},
    numberstyle=\tiny\color{codegray},
    stringstyle=\color{codepurple},
    basicstyle=\ttfamily\footnotesize,
    breakatwhitespace=false,
    breaklines=true,
    captionpos=b,
    keepspaces=true,
    numbers=left,
    numbersep=5pt,
    showspaces=false,
    showstringspaces=false,
    showtabs=false,
    tabsize=2
}

\lstset{style=mystyle}

% \journal{Nuclear Physics B}

\begin{document}

\begin{frontmatter}

%% Title, authors and addresses

%% use the tnoteref command within \title for footnotes;
%% use the tnotetext command for theassociated footnote;
%% use the fnref command within \author or \address for footnotes;
%% use the fntext command for theassociated footnote;
%% use the corref command within \author for corresponding author footnotes;
%% use the cortext command for theassociated footnote;
%% use the ead command for the email address,
%% and the form \ead[url] for the home page:
%% \title{Title\tnoteref{label1}}
%% \tnotetext[label1]{}
%% \author{Name\corref{cor1}\fnref{label2}}
%% \ead{email address}
%% \ead[url]{home page}
%% \fntext[label2]{}
%% \cortext[cor1]{}
%% \affiliation{organization={},
%%             addressline={},
%%             city={},
%%             postcode={},
%%             state={},
%%             country={}}
%% \fntext[label3]{}

\title{Implementação de Modelo de Reservatório Bidimensional Incompressível Integrado com Facilidades de Produção\tnoteref{label_title}}
\tnotetext[label_title]{Relatório final da disciplina PP590A: Tópicos em Geoengenharia de Reservatórios.}

%% use optional labels to link authors explicitly to addresses:
%% \author[label1,label2]{}
%% \affiliation[label1]{organization={},
%%             addressline={},
%%             city={},
%%             postcode={},
%%             state={},
%%             country={}}
%%
%% \affiliation[label2]{organization={},
%%             addressline={},
%%             city={},
%%             postcode={},
%%             state={},
%%             country={}}

\author{Tiago C. A. Amorim\fnref{label_author}}
\tnotetext[label_author]{Atualmente cursando doutorado no Departamento de Engenharia de Petróleo da Faculdade de Engenharia Mecânica da UNICAMP (Campinas/SP, Brasil).}
\ead{t100675@dac.unicamp.br}
\affiliation[Tiago C. A. Amorim]{organization={Petrobras},%Department and Organization
addressline={Av. Henrique Valadares, 28},
city={Rio de Janeiro},
postcode={20231-030},
state={RJ},
country={Brasil}}

\begin{abstract}


\end{abstract}


%%Graphical abstract
% \begin{graphicalabstract}
%\includegraphics{grabs}
% \end{graphicalabstract}

%%Research highlights
% \begin{highlights}
% \item Research highlight 1
% \item Research highlight 2
% \end{highlights}

\begin{keyword}
    Fluxo em Meio Poroso \sep Modelagem Integrada
%% keywords here, in the form: keyword \sep keyword

%% PACS codes here, in the form: \PACS code \sep code

%% MSC codes here, in the form: \MSC code \sep code
%% or \MSC[2008] code \sep code (2000 is the default)

\end{keyword}

\end{frontmatter}

%% \linenumbers

%% main text
\section{Introdução}

    Ao longo da disciplina de Modelagem Integrada foram abordados diferentes tópicos associados à modelagem de reservatórios e o sistema de produção: descrição de fluidos \emph{Black-oil}, fluxo em meios porosos, fluxo vertical multifásico, sistema de \emph{boosting}, problemas de garantia de escoamento, equipamentos de superfície e emissão de $CO_2$.
    Foram desenvolvidos diversos módulos em Python para modelar numericamente o comportamento de alguns dos mecanismos abordados. Este relatório descreve as funcionalidades implementadas e como foi realizada a integração destes módulos para chegar ao modelo integrado proposto.

\section{Implementação}

    Diversos módulos de cálculo foram criados. Cada um com foco em um dos assuntos abordados nas aulas. Buscou-se deixar os módulos como objetos independentes, e assim facilitar o desenvolvimento de novas funcionalidades.

    \subsection{Propriedades de Fluido}

        A descrição do fluido foi implementada como um objeto no módulo \emph{pvt.py}. Foram utilizadas as correlações de Standing\cite{standing1952volumetric} para estimativa de parâmetros de óleo e gás\footnote{As correlações utilizadas foram suprimidas deste relatório para não sobrecarregá-lo de informações, e podem ser encontradas em \cite{rosa2006engenharia}.}:

        \begin{itemize}
            \item Fator volume de formação ($Bg$ e $Bo$).
            \item Razão de solubilidade do óleo ($Rs$).
            \item Viscosidade ($\mu_g$ e $\mu_o$).
        \end{itemize}

        As propriedades são basicamente função da densidade do óleo ($^oAPI$), densidade do gás ($dg$), razão gás-óleo ($RGO$), pressão e temperatura. Outras grandezes auxiliares, necessárias nas estimativas das grandezas de interesse, são calculadas com correlações de Standing (eg.: compressibilidade do óleo).

        As propriedades da água são assumidas constantes e são diretamente fornecidas pelo usuário.

        Este módulo permite ao usuário fornecer um valor de fração de água ($wcut$) e pedir valores de densidade média ($\rho$) e viscosidade média ($\mu$). Os valores são médias ponderadas pela fração de água.

        Também existe a possibilidade de solicitar o uso da correlação de Rønningsen para estimar a viscosidade da mistura água-óleo considerando a formação de emulsões\cite{doi:10.1021/ef00041a001}:

        \begin{align}
            \ln \eta_r &= k_1 + k_2 \, T + k_3 \, wcut  + k_4 \, T \, wcut
        \end{align}

        Onde $\eta_r$ é a razão entre a viscosidade da emulsão e a viscosidade do óleo, $wcut$ é a fração de água e $T$ é a temperatura (em $^oC$). Os valores de $k_i$ são função da taxa de cisalhamento do fluido (Tabela \ref{tab:ronningsen}):

        \begin{align}
            \dot{\gamma} &= \frac{32 Q}{\pi d^3}
        \end{align}

        \begin{table}[ht]
            \centering
            \begin{tabular}{c|ccc}
                T.Cis. ($s^{-1}$) & 30 & 100 & 500 \\
                 \hline
                 $k_1$ & 0.01334 & 0.0412 & -0.06671 \\
                 $k_2$ & -0.003801 & -0.002605 & -0.000775 \\
                 $k_3$ & 4.338 & 3.841 & 3.484 \\
                 $k_4$ & 0.02698 & 0.02497 & 0.005 \\
            \end{tabular}
            \caption{Coeficientes da correlação de Rønningsen.}
            \label{tab:ronningsen}
        \end{table}

        Um pequeno ajuste foi realizado na formulação de Rønningsen. Para manter coerência com o cenário em que existe 100\% de óleo ($wcut=0$), o valor de $\eta_r$ é limitado à unidade (ver trecho reto para valores baixos de $wcut$ nas curvas da Figura \ref{fig:emulsao}).

        \begin{figure}[hbt!]
            \begin{tikzpicture}
                \begin{axis}[
                    grid=both,
                    xlabel = {Corte de água},
                    ylabel = {$\mu_{emuls\tilde{a}o}$ [cP]},
                    legend style={at={(0.90,0.85)}, anchor=east, font=\footnotesize},
                    ]
                    \addplot[color=blue, solid, thick] table [x=WCUT, y=U_1cP] {emulsao.txt};
                    \addplot[color=black, solid, thick] table [x=WCUT, y=U_10cP] {emulsao.txt};
                    \legend{Óleo 1 cP, Óleo 10 cP};
                \end{axis}
            \end{tikzpicture}
            \caption{Estimativa de viscosidade de emulsão a 50 $^oC$.}
            \label{fig:emulsao}
        \end{figure}

        A inversão de fases é estimada com a correlação de Arirachakaran\cite{10.2118/18836-MS}:

        \begin{align}
            \varepsilon_w &= \max \left(0.15; 0.5 - 0.1108 \log_{10} \frac{\mu_o}{\mu_{w,ref}}\right)
        \end{align}

        Onde $\varepsilon_w$ é o valor de fração de água a partir da qual há inversão de fase e $\mu_{w,ref}$ é uma viscosidade de água de referência, igual a 1 cP. Para valores de fração de água maiores que $\varepsilon_w$ a viscosidade da mistura água-óleo é a média ponderada pela fração de água dos respectivos valores.

    \subsection{Fluxo em Meio Poroso}

        As equações que regem o problema de fluxo em meio poroso derivam fundamentalmente da equação de conservação de massa \ref{eq:consmassa} e da equação de fluxo em meio poroso \ref{eq:darcy}, a lei de Darcy \cite{dake1983fundamentals}.

        \begin{align}
            &\sum_{p} \nabla \cdot  (y_{cp} \rho_p v_p) + \sum_{p} (y_{cp} \rho_p q_p) + \sum_{p} \frac{\partial}{\partial t} \left( \phi y_{cp} \rho_p S_p\right) = 0 \label{eq:consmassa} \\
            &v_p = - \frac{k k_{rp}}{\mu_p} \left( \frac{\partial p_p}{\partial x} - \gamma_p \frac{\partial D}{\partial x} \right) = - \frac{k k_{rp}}{\mu_p} \left( \frac{\partial \Phi_p}{\partial x} \right)\label{eq:darcy}
        \end{align}

        A maioria dos simuladores de fluxo comerciais utilizam diferenças finitas \cite{computer2022cmg}\cite{schlumberger2009technical}. Podemos acoplar as equações \ref{eq:consmassa} e \ref{eq:darcy}, e aplicar uma discretização no tempo ($\Delta t$) e no espaço ($\Delta x$) para um volume de controle ($V = \Delta x \Delta y \Delta z$)\footnote{Usualmente chamado de célula.}. A forma unidimensional desta equação discretizada é apresentada a seguir \ref{eq:geralumd}.

        \begin{align}
            \sum_{p} & \left( \frac{\Delta y \Delta z}{\Delta x} y_{cp} \rho_p \frac{k k_{rp}}{\mu_p} \right)_{i+\tfrac{1}{2}} (\Phi_{p,i+1} - \Phi_{p,i})  \nonumber \\
            & + \left( \frac{\Delta y \Delta z}{\Delta x} y_{cp} \rho_p \frac{k k_{rp}}{\mu_p} \right)_{i-\tfrac{1}{2}} (\Phi_{p,i-1} - \Phi_{p,i}) \nonumber \\
            & + q_{cp}^{w} = \frac{1}{\Delta t} \sum_{p} (V \phi y_{cp} \rho_p S_p)_i^{t_i+\Delta t} - (V \phi y_{cp} \rho_p S_p)_i^{t_i} \label{eq:geralumd}
        \end{align}

        Na sua forma mais geral o problema de fluxo em meio poroso precisa ser resolvido para cada um dos componentes que constituem as fases envolvidas. É comum utilizar uma abordagem simplificada, em que poucos componentes são utilizados para representar os fluidos envolvidos. Esta simplificação é usualmente aplicada quando as trocas de fase são \emph{bem comportadas} e passíveis de serem representadas por um conjunto de tabelas, em substituição às equações de estado. Esta abordagem é conhecida como \emph{Black-Oil}.

        As equações também se simplificam ao substituir as densidades ($\rho_p$) e concentrações de componentes ($y_{cp}$) pelo fator volume de formação da fase ($B$) e as relações volumétricas entre as fases e os componentes ($R$)\footnote{Maiores detalhes em \cite{dake1983fundamentals}.}.

        Quando apenas água e óleo estão envolvidos na simulação, apenas duas equações de conservação de massa são necessárias. Como a soma das saturações é igual à unidade ($S_w + S_o = 1$) e negligenciando a tensão interfacial entre a água e o óleo ($p_o = p_w = p$), para cada célula é preciso resolver apenas duas variáveis: $p$ e $S_w$:

        \begin{align}
            &\left( \frac{\Delta y \Delta z}{\Delta x} \lambda_w \right)_{i+\tfrac{1}{2}} (p_{i+1} - p_{i} - \gamma_w \Delta D_{i+\tfrac{1}{2}})  \nonumber \\
            &+ \left( \frac{\Delta y \Delta z}{\Delta x} \lambda_w \right)_{i-\tfrac{1}{2}} (p_{i-1} - p_{i} - \gamma_w \Delta D_{i-\tfrac{1}{2}}) \nonumber \\
            &  = \frac{1}{\Delta t} \left(\frac{V \phi S_w}{B_w}\right)_i^{t_i+\Delta t} - \left(\frac{V \phi S_w}{B_w}\right)_i^{t_i} + q^{std}_w \label{eq:blackoilumdw} \\
            &\left( \frac{\Delta y \Delta z}{\Delta x} \lambda_o \right)_{i+\tfrac{1}{2}} (p_{i+1} - p_{i} - \gamma_o \Delta D_{i+\tfrac{1}{2}})  \nonumber \\
            &+ \left( \frac{\Delta y \Delta z}{\Delta x} \lambda_o \right)_{i-\tfrac{1}{2}} (p_{i-1} - p_{i} - \gamma_o \Delta D_{i-\tfrac{1}{2}}) \nonumber \\
            &  = \frac{1}{\Delta t} \left(\frac{V \phi (1-S_w)}{B_o}\right)_i^{t_i+\Delta t} - \left(\frac{V \phi (1-S_w)}{B_o}\right)_i^{t_i} + q^{std}_o \label{eq:blackoilumdo}
        \end{align}

        \noindent com:
        \begin{align}
            \lambda_p = \frac{k k_{rp}}{B_p \mu_p} \nonumber
        \end{align}

        Observa-se que a pressão ($p$) e a saturação de água ($S_w$) da i-ésima célula tem influência apenas na própria célula e nas células vizinhas. Desta forma, a matriz dos coeficientes deste problema tem termos não nulos em três \emph{bandas} ao redor da diagonal principal. Para modelos bi e tridimensionais são adicionados novos termos fora da diagonal principal.

        No módulo \emph{reservoir.py} foi implementado o problema de fluxo em meio poroso bidimensional, incompressível e plano. Existem poços apenas em duas posições: injetor de água controlado por vazão especificada na célula [1,1] e produtor controlado por pressão de fundo na célula [$n_i$,$n_j$].

        As propriedades de fluido ($Bo$, $Bw$, $\mu_o$ e $\mu_w$) são constantes. O meio poroso também é incompressível, mas é possível definir valores de propriedade de rocha ($\phi$ e $k$) por célula.

        Foram implementadas funções de permeabilidade relativa com as correlações de Corey\cite{1570009749409873792}. As permeabilidades relativas foram implementadas como um objeto à parte no módulo \emph{relative\_permeability.py}.

        Em uma simulação \emph{Black-Oil} com água e óleo, a simulação de fluxo parte de um conjunto de valores de $p$ e $S_w$ no tempo inicial, para cada uma das células, e uma série de controles de produção ($q^{std}_p$). O simulador precisa resolver o conjunto de equações não lineares \ref{eq:blackoilumdw} e \ref{eq:blackoilumdo} a cada passo de tempo.

        Foi implementada a formulação implícita para resolver o problema de fluxo em meio poroso. A implementação mais conhecida para resolver este conjunto de equações não lineares é o Método de Newton-Raphson. Para sistemas incompressíveis (onde $\phi$, $B$ e $\mu$ são constantes) este conjunto de equações é \emph{fracamente} não linear, e apenas os valores de $k_{ro}$ e $k_{rw}$ são função das variáveis de estado. O conjunto de equações resultantes pode ser resolvido adequadamente com o Método do Ponto Fixo\cite{burden2016analise}. As equações são resolvidas como um conjunto de equações lineares de modo iterativo, isto é, utilizando o resultado de uma iteração como dado de entrada para resolver os termos $k_{ro}$ e $k_{rw}$ da próxima iteração, até que a convergência é atingida.

        O controle do tamanho tamanho do passo de tempo de cada iteração é feito de forma simplificada. O usuário precisa fornecer valores de máxima variação de saturação de água e de pressão entre os passos de tempo. Quando algum dos limites não é observado, o passo de tempo é reduzido à metade e a simulação do tempo \emph{atual} é feita novamente. Caso as máximas variações sejam atendidas, o simulador irá incrementar o passo de tempo atual em 20\%. O usuário também pode informar valores mínimo e máximo dos passos de tempo.

        Foi observado que este sistema incompressível tem dificuldades de convergência no primeiro passo de tempo da simulação. O problema está associado ao fato de que a célula em que o injetor de água está conectado tem saturação igual à saturação de água conata. Nesta situação a permeabilidade relativa da água é nula. Como o meio poroso é incompressível, nos primeiros passos de tempo é comum serem observados valores muito alto de pressão nesta célula e nas vizinhas. Este efeito é maior em modelos com células menores e altas vazões de injeção de água.
        Para evitar este efeito indesejado, foi implementada a opção de incrementar a saturação de água no tempo inicial desta célula. Esta alteração tem pouco impacto nos demais resultados, e evita valores de pressão anormais nos primeiros passos de tempo (Figura \ref{fig:ajustesw}).

        \begin{figure}[hbt!]
            \begin{tikzpicture}
                \begin{axis}[
                    grid=both,
                    xlabel = {Tempo [d]},
                    ylabel = {Pwf injetor [bar]},
                    legend style={at={(0.90,0.75)}, anchor=east, font=\footnotesize},
                    ymax=2100.,
                    ymin=900.,
                    xmax=42.,
                    xmin=-2.,
                    ]
                    \addplot[color=black, solid, thick] table [x=Time, y=PwfInj] {SensPwfInjW_0.txt};
                    \addplot[color=blue, solid, thick] table [x=Time, y=PwfInj] {SensPwfInjW_1.txt};
                    \addplot[color=red, solid, thick] table [x=Time, y=PwfInj] {SensPwfInjW_2.txt};
                    \addplot[color=green, solid, thick] table [x=Time, y=PwfInj] {SensPwfInjW_3.txt};
                    \addplot[color=yellow, solid, thick] table [x=Time, y=PwfInj] {SensPwfInjW_4.txt};
                    \addplot[color=gray, solid, thick] table [x=Time, y=PwfInj] {SensPwfInjW_5.txt};
                    \legend{$\Delta S_w=0\%$,$\Delta S_w=2\%$,$\Delta S_w=2\%$,$\Delta S_w=3\%$,$\Delta S_w=4\%$,$\Delta S_w=5\%$};
                \end{axis}
            \end{tikzpicture}
            \caption{Efeito de modificar a saturação inicial de água da célula do injetor de água para um mesmo modelo.}
            \label{fig:ajustesw}
        \end{figure}

    \subsection{Fluxo Vertical Multifásico}

        Forma implementados quatro objetos no módulo \emph{flow.py}. Os cálculos de fluxo vertical multifásico são realizados no elemento mais básico, que representa um trecho de tubulação, com diâmetro e inclinação constante. Este elemento utiliza o módulo \emph{pvt.py} para estimar as propriedades de fluido. Foram consideras as formas de balanço de massa e energia de fluidos quasi-incompressíveis.

        A perda de pressão no elemento pode ser calculada em ambas direções, i.e., alimentando as condições de entrada ($p_a$: pressão de entrada) e calculando as condições de saída ($p_b$: pressão de saída), ou o inverso:

        \begin{align}
            p_a - p_b = \rho g \left( (z_a - z_b) + H_L + \frac{v^2_b - v^2_a}{2 g}\right)
        \end{align}

        O termo de perda de carga por fricção ($H_L$) utiliza a formulação de Colebrook-White para cálculo do fator de fricção ($f$) quando o número de Reynolds ($Re$) é maior que 4000. Para evitar descontinuidades, nos casos em que o número de Reynolds fica entre o fluxo laminar ($Re \leq 2100$) e o fluxo turbulento ($Re \geq 4000$), é aplicada uma correlação linear entre os valores de $f$ calculados para fluxo laminar e turbulento.

        \begin{figure}[hbt!]
            \begin{tikzpicture}
                \begin{semilogxaxis}[
                    grid=both,
                    xlabel = {Número de Reynolds},
                    ylabel = {$f$},
                    legend style={at={(0.90,0.85)}, anchor=east, font=\footnotesize},
                    ]
                    \addplot[color=black, solid, smooth, thick] table [x=Reynold, y=f] {friccao_laminar.txt};
                    \addplot[color=blue, solid, smooth, thick] table [x=Reynold, y=f] {friccao_turbulento.txt};
                    \addplot[color=red, solid, smooth, thick] table [x=Reynold, y=f] {friccao_transicao.txt};
                    \addplot[color=black, dashed, smooth, thick] table [x=Reynold, y=f] {friccao_laminar_trans.txt};
                    \addplot[color=blue, dashed, smooth, thick] table [x=Reynold, y=f] {friccao_turbulento_trans.txt};
                    \legend{Laminar, Turbulento, Transição};
                \end{semilogxaxis}
            \end{tikzpicture}
            \caption{Proposta de cálculo do fator de fricção na região de transição.}
            \label{fig:friccao}
        \end{figure}

        Não foram implementadas correlações para troca de temperatura. Todo o cálculo é realizado assumindo temperatura constante por elemento. Também não foi implementado o cálculo de fluxo com gás livre, de modo que apenas água e óleo podem existir no sistema.

        O termo de ganho de energia com turbomáquinas foi deliberadamente retirado porque foi criado um objeto específico para representar uma bomba elétrica submersível (\emph{ESP}). Não foram implementadas tabelas de eficiências de bomba. O objeto \emph{ESP} apenas recebe um valor de incremento de pressão e um valor de eficiência. O incremento de pressão é adicionado nos cálculos do fluxo vertical multifásico no ponto em que o usuário colocar a bomba. Um segundo comando realiza os cálculos da potência necessária para aplicar o incremento de pressão ao fluido em questão.

        O terceiro objeto implementado é uma coleção de trechos de tubulações. Desta forma é possível descrever uma tubulação e discretizá-la para fazer os cálculos de fluxo vertical multifásico.

        O último objeto é uma coleção de tubulações e bombas. Neste elemento é possível descrever todo o conjunto de elementos que interligam o poço à unidade de produção: colunas, bombas, linhas, \emph{risers} (exemplo na Figura \ref{fig:flow}). É também neste objeto que é possível fazer o cálculo do ponto de operação de um poço, seja com uma IPR ou com um modelo de fluxo. É utilizado o Método da Secante para encontrar o ponto de operação.

        \begin{figure}[hbt!]
            \begin{tikzpicture}
                \begin{axis}[
                    grid=both,
                    xlabel = {Comprimento de Linha [m]},
                    ylabel = {Pressão [bar]},
                    legend style={at={(0.90,0.85)}, anchor=east, font=\footnotesize},
                    ]
                    \addplot[color=black, solid, thick] table [x=Length_m, y=Pressure_bar] {flow_ex_20pc.txt};
                \end{axis}
            \end{tikzpicture}
            \caption{Exemplo de resultado de simulação de fluxo vertical multifásico do reservatório à unidade de produção, composto por coluna, \emph{ESP}, linha horizontal e \emph{riser}.}
            \label{fig:flow}
        \end{figure}

    \subsection{Modelo de Reservatório Proposto}


    \subsection{Modelo de Reservatório Proposto}
        O modelo de reservatório é plano, com forma quadrada e mesmo refinamento em ambas direções ($\Delta i = \Delta j$). Existe um injetor de água em um dos cantos (célula [1,1]) e um produtor na diagonal oposta (célula [n,n]). As propriedades de rocha ($\phi$ e $k$) são constantes.

        Em uma simulação \emph{Black-Oil} com água e óleo, a simulação de fluxo parte de um conjunto de valores de $p$ e $S_w$ no tempo inicial, para cada uma das células, e uma série de controles de produção ($q^{std}_p$). O simulador precisa resolver o conjunto de equações não lineares \ref{eq:blackoilumdw} e \ref{eq:blackoilumdo} a cada passo de tempo.

        A implementação mais conhecida para resolver este conjunto de equações não lineares é o Método de Newton-Raphson. Para sistemas incompressíveis (onde $\phi$, $B$ e $\mu$ são constantes) este conjunto de equações é \emph{fracamente} não linear, e apenas os valores de $k_{ro}$ e $k_{rw}$ são função das variáveis de estado. O conjunto de equações resultantes pode ser resolvido adequadamente com o Método do Ponto Fixo. As equações são resolvidas como um conjunto de equações lineares de modo iterativo, isto é, utilizando o resultado de uma iteração como dado de entrada para resolver os termos $k_{ro}$ e $k_{rw}$ da próxima iteração, até que a convergência é atingida.

        A proposta deste trabalho é avaliar a performance do Método de Eliminação de Gauss com Pivotamento Parcial com Escala. A partir de um código de fluxo em meio poroso bidimensional e incompressível foram gerados arquivos CSV com as matrizes de coeficientes e vetores de constantes para a resolução de uma iteração do Método do Ponto Fixo, para diferentes refinamentos de malha.

        % Também foram exportados os resultados de cada uma das respostas destes sistemas de equações. No simulador de fluxo a resolução do sistema de equações foi feito com o pacote Numpy. Segundo a documentação do pacote, são utilizadas rotinas do LAPACK \cite{dongarra1992lapack}.

\section{Implementação} \label{sec:implementacao}

        Todo o código utilizado nesta análise foi desenvolvido em C++. A duas funções principais são:

        \begin{description}
            \item[readCSV] Função que recebe um \emph{string} com o camimnho de um arquivo CSV e faz a sua leitura. É assumido que é utilizado vírgula como separador. A função retorna uma matrix de \emph{double}.
            \item[SolveGauss] Função que recebe uma matrix de \emph{double} e um vetor de \emph{double}, e resolve o sistema de equações lineares usando Eliminação de Gauss com Pivotamento Parcial. Existe a opção de realizar o pivotamento com e sem uso de escala.
        \end{description}

\section{Resultados}

        Foram realizadas duas avaliações. A primeira foi da qualidade das respostas encontradas com o Método de Eliminação de Gauss. As respostas da implementação do Método de Eliminação de Gauss foram comparadas com as respostas do método de solução de sistema de equações lineares utilizado no simulador de fluxo em meio poroso (do pacote Numpy). Foram feitos testes com as duas variações do Método de Eliminação de Gauss: sem e com uso de escala.

        A medida da qualidade das respostas de cada método foi baseada no residual na forma $R = Ax-B$, onde $A$ é a matriz de coeficientes, $B$ é o vetor de constantes e $x$ é o vetor da solução do sistema linear $Ax=B$. Foram avaliadas duas normas do vetor residual: $L^2$ (\ref{eq:normadois}) e $L^\infty$ (\ref{eq:normainf}).

        \begin{align}
            ||x||_2 &:= \sqrt{\sum_{i}x_i^2} \label{eq:normadois} \\
            ||x||_\infty &:= \max_{i} |x_i| \label{eq:normainf}
        \end{align}

        As figuras \ref{fig:normaeledois} e \ref{fig:normaeleinf} comparam as normas dos residuais da solução dos sistemas de equações propostos com três métodos:
        \begin{itemize}
            \item Eliminação de Gauss com Pivotamento Parcial
            \item Eliminação de Gauss com Pivotamento Parcial com Escala
            \item \emph{Rotinas do Numpy}\footnote{Este residual foi calculado com base nas respostas geradas com as rotinas de cálculo utilizadas no simulador de fluxo. Estes resultados fazem parte desta comparação para representar respostas de \emph{melhor qualidade}.}
        \end{itemize}

        Observa-se que para o problema proposto de resolver sistemas de equações de um modelo de fluxo em meio poroso bidimensional, não há necessidade de usar escala no Método de Eliminação de Gauss. Uma segunda constatação é de que a qualidade das respostas geradas com o Método de Eliminação de Gauss é tão boa quanto a dos métodos numéricos implementados no Numpy, que seguramente são mais sofisticados que os métodos implementados para esta avaliação.

        % \begin{figure}[hbt!]
        %     \begin{tikzpicture}
        %         \begin{semilogyaxis}[
        %             grid=both,
        %             xlabel = {Número de Parâmetros},
        %             ylabel = {$||x||_2$},
        %             legend style={at={(0.90,0.85)}, anchor=east, font=\footnotesize},
        %             ]
        %             \addplot[color=blue, solid, smooth, thick, mark=*] table [x=Parameters, y=NormR] {results_sens_table.txt};
        %             \addplot[color=red, dashed, smooth, thick, mark=o] table [x=Parameters, y=NormRScl] {results_sens_table.txt};
        %             \addplot[color=black, solid, smooth, thick, mark=*] table [x=Parameters, y=NormRTrue] {results_sens_table.txt};
        %             \legend{Elim.Gauss, Elim.Gauss+Esc., \emph{Numpy}};
        %         \end{semilogyaxis}
        %     \end{tikzpicture}
        %     \caption{Norma $L^2$ do residual das soluções de sistemas de equações lineares com diferentes métodos.}
        %     \label{fig:normaeledois}
        % \end{figure}




        O código foi implementado em C++ e em um único arquivo. Pode ser encontrado em \href{https://github.com/TiagoCAAmorim/IntegratedModel/}{https://github.com/ TiagoCAAmorim/IntegratedModel/}.

    \section{Conclusão}

        O Método de Eliminação de Gauss com Pivotamento Parcial com Escala gerou boas respostas para os sistemas de equações lineares testados. O residual das respostas do método implementado foram da mesma ordem de grandeza dos resultados alcançados com um pacote de computação científica popular. O Método de Eliminação de Gauss se mostrou computacionalmente lento, chegando a ser inviável para resolver sistemas de equações com um número maior de variáveis.

    % \label{}

%% The Appendices part is started with the command \appendix;
%% appendix sections are then done as normal sections

\appendix

\section{Lista de Variáveis}

\begin{description}
    \item[$^oAPI$:]Grau API do óleo.
    \item[$Bp$:]Fator volume de formação da fase $p$ no reservatório ($m^3/m^3$).
    \item[$c_o$:]Compressibilidade do óleo ($1/bar$).
    \item[$D$:]Profundidade.
    \item[$d$:]Diâmetro da tubulação ($m$).
    \item[$dp$:]Densidade relativa da fase $p$.
    \item[$\Delta x$:]Discretização espacial na direção $i$.
    \item[$\Delta y$:]Discretização espacial na direção $j$.
    \item[$\Delta z$:]Discretização espacial na direção $k$.
    \item[$\Delta t$:]Discretização temporal.
    \item[$\gamma_p$:]Peso específico da fase $p$ ($\gamma_p = \rho_p g$).
    \item[$\dot{\gamma}$:]Taxa de cisalhamento ($1/s$).
    \item[$k$:]Permeabilidade absoluta do meio poroso.
    \item[$k_{rp}$:]Permeabilidade relativa da fase $p$.
    \item[$\mu_p$:]Viscosidade da fase $p$.
    \item[$p_p$:]Pressão da fase $p$.
    \item[$\phi$:]Porosidade da rocha.
    \item[$q_p$:]Vazão volumétrica da fase $p$.
    \item[$q^w_{cp}$:]Vazão mássica do componente $c$ na fase $p$.
    \item[$q^{std}_{cp}$:]Vazão volumétrica da fase $p$ medida em condições padrão (\emph{standard}).
    \item[$RGO$]Razão gás-óleo em condições de superfície ($m^3/m^3$).
    \item[$\rho_p$:]Densidade da fase $p$.
    \item[$S_p$:]Saturação da fase $p$ no meio poroso.
    \item[$V$:]Volume total do volume de controle (célula).
    \item[$wcut$:]Corte de água ($Qw/(Qw+Qo)$).
    \item[$v_p$:]Velocidade da fase $p$.
    \item[$y_{cp}$:]Concentração do componente $c$ na fase $p$.
\end{description}

%% \section{}
%% \label{}

%% If you have bibdatabase file and want bibtex to generate the
%% bibitems, please use
%%

\bibliographystyle{elsarticle-num}
\bibliography{refs}

%% else use the following coding to input the bibitems directly in the
%% TeX file.

% \begin{thebibliography}{00}

%% \bibitem{label}
%% Text of bibliographic item

% \bibitem{}

% \end{thebibliography}

% \newpage
% \FloatBarrier
% \section{Código em C}

% O código de ambos métodos foi implementado em um único arquivo. O código é apresentado em duas partes neste documento para facilitar a leitura. O código pode ser encontrado em \href{https://github.com/TiagoCAAmorim/numerical-methods}{https://github.com/TiagoCAAmorim/numerical-methods}.

% \subsection{Método da Bissecção}
% \lstinputlisting[language=C, linerange={1-229}]{./02_newton_raphson.c}

% \subsection{Método de Newton-Raphson}
% \lstinputlisting[language=C, linerange={231-445}]{./02_newton_raphson.c}

% \subsection{Método da Mínima Curvatura}
% \lstinputlisting[language=C, linerange={448-958}]{./02_newton_raphson.c}

\end{document}
\endinput